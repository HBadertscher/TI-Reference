\section{Komplexe Zahlen}

Komplexe Zahlen können im TI Voyage 200 in der Form $a+b\imath$ (Rectangular) oder 
$r \: \angle \: \phi$ (Polar) geschrieben werden. Unter 
$MODE \blacktriangleright Complex \ Format$ kann der Standard-Modus ausgewählt werden.
Der Modus $Real$ zeigt nur komplexe Werte an, wenn auch die Eingabe komplex war.
Die Eingabe und Umrechnung geschieht folgendermassen: \\
\begin{tabular}{l l l}
	$2 5+3*\imath$			& | & $... \blacktriangleright Rect$	\\
	$(2 \angle 30) $		& | & $... \blacktriangleright Polar$	\\
\end{tabular}

\subsection{Funktionen}
\begin{tabular}{|l|l|}
	\hline
	$cFactor(x^2+a^2,x)$								& Komplexe Faktorzerlegung nach $x$						\\ \hline
	$cSolve(x^2+x+1,x)$									& Lösen der komplexen Gleichung nach $x$				\\
	$cSolve(x=2*y \: and \: y^2=-1)$					& Lösen komplexer Gleichungssysteme nach $x$ und $y$	\\ \hline
	$cZeros(x^2+1,x)$									& Bestimmen der (komplexen) Nullstellen					\\ \hline
	$conj(z)$											& Konjugiert-komplexe Zahl $\bar{z}$					\\ \hline
	$abs(z)$											& Betrag $|z|$											\\ \hline
	$angle(z)$											& Winkel $\arg(z)$										\\ \hline
	$real(z)$											& Realteil $\Re(z)$										\\ \hline
	$imag(z)$											& Imaginärteil $\Im(z)$									\\ \hline
\end{tabular}

\subsection{Umrechnungen}
\begin{tabular}{|l|l|}
	\hline
	$P \blacktriangleright Rx(r,\theta)$						& Gibt die $X$-Koordinate des Paars $(r,\phi)$ zurück.		\\ 
	$P \blacktriangleright Rx(\{r1,r2\},\{\theta 1,\theta 2\})$	& Funktioniert auch für Listen...							\\ 
	$P \blacktriangleright Rx([r1,r2;r3,r4],[\theta 1,\theta 2; \theta 3, \theta 4 ]) $
																& .. und Vektoren / Matrizen								\\ \hline
	$P \blacktriangleright Ry(r,\phi)$							& Gibt die $Y$-Koordinate des Paars $(r,\phi)$ zurück.		\\ \hline
	$R \blacktriangleright Pr(x,y) $							& Gibt die $r$-Koordinate des Paars $(x,y)$ zurück.			\\ \hline
	$R \blacktriangleright P \theta(x,y)$						& Gibt die $\theta$-Koordinate des Paars $(x,y)$ zurück.	\\ \hline
\end{tabular}