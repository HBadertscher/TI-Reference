\section{Statistik / Wahrscheinlichkeit}

\subsection{Funktionen}
\begin{tabular}{|l|l|}
	\hline
	$mean(\{...\})$				& Berechnet das arithmetische Mittel der Elemente der Liste. \\
	$mean(\{...\},\{...\})$		& Mit einer zweiten Liste lassen sich die Elemente einzeln gewichten. \\ \hline
	$mean(A)$					& Gibt einen Zeilenvektor mit den arith. Mitteln der Spalten zurück. \\ 
	$mean(A,B)$					& Mit einer Matrix $B$ lassen sich die Elemente von $A$ gewichten. \\ \hline
	$median(\{...\})$			& Berechnet den Median der Elemente der Liste. \\
	$median(A)$					& Gibt einen Zeilenvektor mit den Medianwerten der Spalten zurück. \\ \hline
	$stdDev(\{...\})$			& Berechnet die Standardabweichung $\sigma$ der Liste \\ \hline
	$variance(\{...\}$			& Berechnet die Varianz $\sigma ^2$ der Liste \\ \hline
	$nCr(n,k)$					& Binominalkoeffizient $\binom{n}{k}$ - funktioniert auch für Listen und Matrizen	\\ \hline
	$nPr(n,k)$					& Anzahl Möglichkeiten unter Berücksichtigung der Reihenfolge $k$ Elemente aus $n$ auszuwählen. \\ \hline
	$OneVar L1,[L2],[L3],[L4]$	& Berechnet die Statistiken der Liste L1. Die Statistik wird mit $ShowStat$ eingeblendet. \\
	$ShowStat$					& Folgende Werte werden berechnet: $ \bar x, \: \sum x, \: \sum x^2, \: \sigma x, \: ...$ \\
								& Optionale Listen: $L2$: Häufigkeit, $L3$: Klassencodes, $L4$: Klassenliste \\ \hline
	$TwoVar L1,L2,[L3],[L4],[L5]$ & Gleich wie $OneVar$, einfach für 2 Variablen. \\
	$ShowStat$ & $L1$, $L2$: Variablen $X$ und $Y$, $L3-5$: wie bei $OneVar$ \\ \hline
\end{tabular}

\subsection{Regression}
Zur Berechnung einer Regression muss eine Liste ($\{...\}$) die x-Werte enthalten 
und eine zweite Liste die y-Werte. Der Befehl $LinReg \: L1,L2$ berechnet die lineare 
Regression. Mit $ShowStat$ werden die berechneten Werte angezeigt. 
Es ist auch möglich, die Datenpunkte und die Regressionskurve zu plotten: 
$Regeq(x) \to y1(x)$ und $NewPlot \: 1,1,L1,L2$ Optional können weitere Listen 
angegeben werden: L3: Häufigkeit, L4: Klassencodes,  L5: Klassenliste, wobei alle Listen ausser 
L5 die gleiche Dimension besitzen müssen. 'Iterationen' gibt die maximale Anzahl Lösungsversuche an. 
(standardmäsig: 64)\\

\begin{tabular}{|l|l|}
	\hline
	Lineare Regression						&	$LinReg \: L1,L2,[L3],[L4,L5]$ \\ \hline
	Logarithmische Regression				&	$LnReg \: L1,L2,[L3],[L4,L5]$ \\ \hline
	Logistische Regression					&	$Logistic \: L1,L2,[\text{Iterationen}],[L3],[L4,L5]$\\ \hline
	Potenz-Regression						&	$PowerReg \: L1,L2,[L3],[L4,L5]$ \\ \hline
	Quadratische Polynomische Regression	&	$QuadReg \: L1,L2,[L3],[L4,L5]$ \\ \hline
	Kubische Regression						&	$CubReg \: L1,L2,[L3],[L4,L5]$ \\ \hline
	Polynomische Regression 4-ter Ordnung	&	$QuartReg \: L1,L2,[L3],[L4,L5]$ \\ \hline
\end{tabular}

\subsection{Zufallszahlen}
\begin{tabular}{|l|l|}
	\hline
	$RandSeed \: 1147$					& Setzt die Ausgangsbasis (Seed) für den Zufallszahl-Generator \\ \hline
	$rand()$							& Gibt eine Zufallszahl zwischen $0$ und $1$ zurück. \\
	$rand(n)$							& Gibt eine Zufallszahl zwischen $0$ und $n$ (für $n$ pos.) \\ 
										& bzw. zwischen $n$ und $0$ (für $n$ neg.) zurück. \\ \hline
	$randMat(n,m)$						& Erzeugt eine ganzzahlige Matrix mit $n$ Zeilen und $m$ Spalten mit Werten $-9<x<+9$. \\ \hline
	$randNorm(a,sd)$					& Gibt eine reelle Zufallszahl um den Mittelwert $a$ mit der Standardabweichung $sd$ aus. \\ \hline
	$randPoly(x,n)$						& Erzeugt ein Polynom der Variable $x$ der Ordnung $n$ mit Koeffizienten $-9<x<+9$ \\ \hline
\end{tabular}