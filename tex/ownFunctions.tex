\section{Eigene Funktionen}

Eigene Funktionen werden mit dem $Define$ Befehl definiert. Die Syntax lautet wie folgt: \\
$ Define \: func(x) = y$

Stückweise definierte Funktionen können mit dem $when$ Befehl unterschieden werden: \\
$ Define \: func(x)=when(Bedingung,Dann,Sonst) $ \\
Auch Verschachtelungen sind möglich: \\
$ Define \: func(x)=when(Bedingung1,Dann,when(Bedingung2,Dann,Sonst)) $

\subsection{Praktische Funktionen}
	\begin{tabular}{|l|l|}
		\hline
		Sprungfunktion & $Define \: sp(x)=when(x<0,0,when(x=0,1/2,1)) $\\ \hline
		Signum & $Define \: sgn(x)=when(t<0,-1,when(x=0,0,1))$ \\ \hline
		Rechteckimpuls & $Define \: pa(x,a)=sp(x+a)-sp(x-a)$ \\ \hline
	\end{tabular}
